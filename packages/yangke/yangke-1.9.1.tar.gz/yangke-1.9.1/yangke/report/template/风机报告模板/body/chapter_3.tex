% !Mode:: "TeX:UTF-8" 

\BiChapter{提高火电机组灵活性的方法}{}

\BiSection{提高锅炉灵活性}{arg2}
对于火电机组,在燃烧系统方面可以通过优化电厂磨煤机和燃烧器的协同配合、优化燃烧器降低最小燃烧功率、加装煤粉仓等技术措施提高机组灵活性。通过以下途径实现锅炉的灵活性改造。

\begin{enumerate}
	\renewcommand{\labelenumi}{(\theenumi)}
	\item 诊断试验:找出限制机组深度调峰和稳定运行的瓶颈,挖掘锅炉最低稳燃能力和机组负荷提升能力,为灵活性运行提供基础数据。
	\item 技术改造:精细化调整,燃烧器改造,制粉系统改造,煤质掺烧,燃烧监控改造,智能控制等
	\item 试验验证:针对改造内容制定试验方案,验证灵活性改造效果。
\end{enumerate}

提高锅炉的灵活性,尽可能使锅炉可以在低负荷下稳定运行,在节省燃料的同时提高火电机组的灵活性。

\begin{table}[!ht]
	\renewcommand{\arraystretch}{1.2}
	\centering\wuhao
	\caption{国内部分燃煤机组锅炉最低稳燃负荷} \label{tab_ch2} \vspace{2mm}
	\begin{tabularx}{\textwidth}{*{4}Y}
	\toprule[1.5pt]
		公司 & 最低负荷 \\
	\midrule[1pt]
		陕西秦岭发电有限公司 & 20\%BMCR \\
		淮浙煤电有限责任公司凤台发电分公司 & 50 \\
		 5 & 40 \\
	\bottomrule[1.5pt]
	\end{tabularx}
\end{table}

\section{提高汽机灵活性}
当锅炉的最低稳燃负荷确定后,还可以在汽轮机系统方面通过高压缸旁路、主蒸汽旁路、高压再热器旁路等技术措施提高机组灵活性。

\section{建立储能系统}
国外众多研究表明,建立储能系统是提高火电机组运行灵活性的一个重要手段,主要包括热能储存(蓄水罐、电锅炉)、燃料电池、电化学储能(储氢)和机械储能(空气压缩、飞轮储能)等。 

\section{燃气蒸汽联合循环}
燃气轮机和蒸汽轮机联合发电机组的灵活性要远远高于常规火电机组,利用燃气轮机烟气驱动蒸汽轮机,蒸汽轮机系统同时实现供电和供热,当电网负荷变化时,通过调节燃气轮机和蒸汽轮机的发电比例实现负荷跟踪。